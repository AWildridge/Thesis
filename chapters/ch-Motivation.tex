\ProvidesFile{chapters/ch-Motivation.tex}

\chapter{Introduction}
\label{ch:introduction}

Particle physics ultimately has one goal which is to understand the underlying structure of our universe. This is a very ambitious goal and one can always wonder if there's some additional structure that we are ignorant of at smaller length scales. For example, we typically refer to electrons unquestionably as fundamental elementary particles meaning they are indivisible and not composed of something more elementary than an electron. However, a particle physicist will tell you that we only know electrons are not composite to X \ref{ref:electron_scale}. Thus, one may conclude that a particle physicist's job is never complete and there may always be that lingering question of ``what if''. I'd like to try and motivate here that there is a more urgent necessity, as of the writing of this thesis, to study particle physics due to our current inability to describe certain phenomena.

There are many examples of interesting phenomena that we currently cannot explain.

\section{Indirect Evidence of Beyond-the-Standard-Model Physics}

\subsection{Mass of Neutrinos}

\subsection{Matter-antimatter asymmetry}

\subsection{Dark Matter}

\subsection{Dark Energy}

\subsection{Gravity}

\section{Beauty Problems}

\subsection{Instability of the Electroweak Vacuum}

\subsection{Strong CP problem}

\subsection{Hierarchy problems}

\subsection{Generation problem}

\subsection{Unification}

\section{Organization of the Thesis}

