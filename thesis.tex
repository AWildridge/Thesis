\ProvidesFile{thesis.tex}[2023-02-03 PurdueThesis thesis.tex file]

%
%  The home page for the PurdueThesis software is
%      https://engineering.purdue.edu/~mark/PurdueThesis/
%
%  Be sure to sign up for the PurdueThesis mailing list at
%      https://engineering.purdue.edu/ECN/mailman/listinfo/purduethesis-list
%  so you learn of new versions of this software.  You must be on that
%  mailing list to receive help with this software.
%
%  This is the template root file for an example thesis (for master's
%  degree) or dissertation (for a Ph.D.).  From now on "thesis" will
%  refer to both of these unless stated otherwise.
%
%  LaTeX systems include auxiliary programs to do bibliographies,
%  indexes, etc.  The latexmk program runs the fewest programs needed
%  to update your thesis.  latexmk runs automatically on Overleaf.  If
%  you're LaTeXing your document on a non-Overleaf system you may need
%  to run latexmk manually.
%
%  This thesis contains Feynman diagrams in the ap-physics.tex file.
%  For these to be processed correctly you must use the lualatex
%  program:
%      latexmk -lualatex thesis
%  (If your thesis doesn't have Feynman diagrams---the
%      \include{ap-physics}
%  command may be commented out by prefixing it with a
%  '%') use pdflatex instead of lualatex:
%      latexmk thesis
%
%  To make a final PDF file before you turn in your thesis do
%      latexmk -g thesis
%  This makes sure than everything is done for your final version.
%
%  References cited below:
%
%  TM2017 is short for Thesis Manual 2017:
%    A Manual for the Preparation of Graduate Theses,
%    eighth revised edition,
%    Thesis and Dissertation Office,
%    Purdue University,
%    2017,
%    revised August 30, 2017,
%    http://www.purdue.edu/gradschool/documents/thesis/graduate-thesis-manual.pdf,
%    last retrieved on May, 8, 2021.
%
%  In this file, change the example information to your information.
%

% institution
% Choose an institution name from the following list:
%     VALUE                   COMMENT
%     Purdue University
\def\ZZinstitution{Purdue University}


% campus
% Choose a campus from the following list:
%     VALUE               COMMENT
%     West\space Lafayette
\def\ZZcampus{West\space Lafayette}


% program
% Choose a program from the following list:
%     VALUE
%     Physics
%     Physics and Astronomy
\def\ZZprogram{Physics and Astronomy}

% degree
% Choose a degree from the following list:
%     Doctor of Philosophy
\def\ZZdegree{Doctor of Philosophy}

% author
% Put your name here.
\def\ZZauthor{Name Here}

% document
% Choose a document from the following list:
%     A Dissertation
%     A Master's Bypass Report
%     A Preliminary Report
%     A Thesis
\def\ZZdocument{A Dissertation}

% graduation
% Chose a month from
%     May
%     August
%     December
% followed by a space
% then choose a year from 2020 to 2030.
\def\ZZgraduation{December 2023}

% title
% If you need to manually split the title,
% over several lines do, for example,
%     \def\ZZtitle{%
%       This is the First Line\\[-6pt]
%       and this is the Second Line%
%     }
\def\ZZtitle{Title Here}


% showcolophon
% Print the ap-colophon.tex file at the end of the document?
% THE SUBMITTED COPY OF YOUR THESIS MUST BE RUN WITH ZZshowcolophon = {false}.
\def\ZZshowcolophon{false}

% showdiagonalline
% Show a diagonal line from lower left to center
% of main printed part of page?
% THE SUBMITTED COPY OF YOUR THESIS MUST BE RUN WITH ZZshowdiagonalline = {false}.
\def\ZZshowdiagonalline{false}

% showgridlines
% Show grid lines on main printed part of page
% Vertical and horizontal grid lines are put
% in the normal printed part of the page---this
% includes lines where the margins are.
% THE SUBMITTED COPY OF YOUR THESIS MUST BE RUN WITH ZZshowgridlines = {false}.
\def\ZZshowgridlines{false}

% showmarginlines
% Show margin lines on the edge of the normal printed part of the page?
% Margin lines show where the margins are.
% THE SUBMITTED COPY OF YOUR THESIS MUST BE RUN WITH ZZshowmarginlines = {false}.
%     VALUE    MEANING
%     false    don't show marginlines
%     true     show marginlines
\def\ZZshowmarginlines{false}

% showtimestamp
% Show, for example, a "compiled on  2021-03-02  Tuesday  17:16:24"
% timestamp in the upper right corner of page?
%     VALUE    MEANING
%     false    don't show timestamp
%     true     show timestamp
% THE SUBMITTED COPY OF YOUR THESIS MUST BE RUN WITH ZZshowtimestamp = {false}.
\def\ZZshowtimestamp{false}

% todonotes
% Set things up for todonotes.
%     VALUE    MEANING
%     false    don't put todo notes in PDF file
%     true     put 0.8 inch wide todo notes in PDF file
%     wide     put 3.8 inch wide todo notes in PDF file, do not send
%              todonotes = wide output to a printer
% THE SUBMITTED COPY OF YOUR THESIS MUST BE RUN WITH todonotes = {false}.
\def\ZZtodonotes{false}

% Mark Senn uses an `optional-debugging-code.tex file' but does not
% distribute it.  The following line won't do anything if you don't
% have an optional-debugging-code.tex file so you can leave it the
% way it is.
\InputIfFileExists{optional-debugging-code.tex}{}{}

% The \includeonly command can be used to only include some
% files that have \include commands below.  This is handy
% to only include some files so your document will LaTeX
% faster or if you're trying to narrow down where an error
% occurs.  You can use
%     \includeonly{ch-introduction}
% to only include ch-introduction.tex, or
%     \includeonly{ch-introduction,ap-about-appendices}
% to include ch-introduction.tex and ap-about-appendices.tex.
% More files can be added---just put ',' between the names.
% Comment out the following line before submitting the
% final copy of your thesis.
% \includeonly{ch-introduction,ap-about-appendices}

\documentclass{PurdueThesis}


%%%% \ExplSyntaxOn                         %%%% changed 2021-07-27 by mark
%%%% \bool_set_true:N \ZZCenterCaptionB    %%%% changed 2021-07-27 by mark
%%%% \ExplSyntaxOff                        %%%% changed 2021-07-27 by mark

\def\ZZatinformation{}
% If you are at the Hammond or Westville campus
% remove the "%" from the begining of the next line.
%\def\ZZatinformation{~at~Purdue~Northwest}

% If the title contains commas, do, for example,
% \def\ZZtitle{WIRELESS POWER TRANSFER:
% EFFICIENCY, FAR FIELD, DIRECTIVITY, AND PHASED ARRAY ANTENNAS}



% PurdueThesis.cls loads the rotating package which loads the graphicx
% package.  From page 12 of "Packages in the `graphics' bundle", 2021-03-05,
% retrieved 2021-06-16, at https://texdoc.org/serve/grfguide.pdf/0
%     \graphicspath{<dir-list>}
%
%         This optional declaration may be used to specify a list of
%         directories in which to search for graphics files.  The
%         format is the same as for the LaTeX 2e primitive \input@path.
%         A list of directories, each in a {} group (even if there is
%         only one in the list).  For example:
%             \graphicspath{{eps/}{tiff/}}
%         would cause the system to look in the subdirectories eps and
%         tiff of the current directory.  (All modern TeX systems use /
%         as the directory separator, even on Windows.)
%
%         The default setting of this path is \input@path that is:
%         graphics files will be found whereever TeX files are found.
%
% Look in the "graphics" subfolder for graphics files.
% This is done to reduce the number of files in the main thesis folder
% so the ones in there are easier to find.
\graphicspath{{graphics/}}

% Look in the "packages" subfolder for packages.
% This is done to reduce the number of files in the main thesis folder
% so the ones in there are easier to find.
\makeatletter
  \def\input@path{{packages/}}
\makeatother

%
% Configure bibliography.
%
% Automatically configure the bibliography.  Based on the
% institution, campus, and program listed in the \documentclass
% command \ZZBibProcessor is set to "BibLaTeX" or "BibTeX".
% For BibLaTeX, a
%    \usepackage[...]{biblatex}
% is done.  Put your bibliography entries in all-biblatex.bib.
% For BibTeX, a
%     \bibliographystyle{...}
% command is done.  Put your bibliography entries in all-bibtex.bib.
%
% All combinations of institution, campus, and program use BibLaTeX.
% Exceptions that use BibTeX:
%     o  "Purdue University", "West Lafayette", "Earth, Atmospheric,
%        and Planetary Sciences" uses the ametsoc2014 bibliography style.
%     o  "Purdue University", "West Lafayette", "Veterinary Clinical
%        Sciences" uses the ama bibliography style.
%
% To override the default choices picked by \ConfigureBibliography, change,
% for example,
%     \ConfigureBibliography
% to
%     % \ConfigureBibliography
%     \newcommand{\ZZBibProcessor}{BibLaTeX}
%     \usepackage[backend=biber, citestyle=apa, dashed=false, sortcites=true, style=apa]{biblatex}
%     \addbibresource{all-biblatex.bib}
\ConfigureBibliography

%
% This is only done if you are using BibLaTeX.
%
%
% If you don't want to ignore urldate fields,
% comment out (put "%" before) the next ten lines.
%
\DeclareSourcemap
  {
    \maps[datatype=bibtex]
    {
      % Ignore "urldate = {...}" in .bib files.
      % See the first complete example on page 201 of
      %     https://mirrors.rit.edu/CTAN/macros/latex/contrib/biblatex/doc/biblatex.pdf
      \map
        {
          \step[fieldset=urldate, null]
        }
        % Enter approximate (circa) dates using, for example,
        % "year = c2020"  See
        %     https://tex.stackexchange.com/questions/224617/what-is-the-correct-way-to-handle-approximate-dates-in-biblatex
      \map[overwrite=false]
        {
          \step[fieldsource=year]
          \step[fieldset=sortyear, origfieldval, final]
          \step[fieldsource=sortyear, match={c}, replace={}]
        }
    }
  }

% To let {\bfseries\scshape text} work as expected.
% See
%     https://tex.stackexchange.com/questions/27411/small-caps-and-bold-face
\usepackage{bold-extra}

% For chemical figures.
\usepackage{chemfig}

% For typesetting cryptography pseudocode, algorithms, and protocols.
% See
%     https://mirror.las.iastate.edu/tex-archive/macros/latex/contrib/cryptocode/cryptocode.pdf
\usepackage
[
  n,            % or lambda
  advantage,
  operators,
  sets,
  adversary,
  landau,
  probability,
  notions,
  logic,
  ff,
  mm,
  primitives,
  events,
  complexity,
  oracles,
  asymptotics,
  keys,
]
{cryptocode}

% Define
%    \VerbatimInput[options]{filename}
%    \begin{VerbatimOut}{filename} ... \end{VerbatimOut}.
\usepackage{fancyvrb}
  \DefineShortVerb{\|}  % so "|verbatim|" will be verbatim

% For \InpuutIfFileExists.
\usepackage{filehook}

% So "_" will work in URLs when using BibTeX.
\usepackage[T1]{fontenc}

% For nlui testing.
\usepackage{listings}

% For chemical equations.
% See
%     https://ctan.org/pkg/mhchem?lang=en
% From the "Package documentation" linked-to document
%     mhchem needs a couple of other packages.
%     For instance, expl3, amsmath and calc.
\usepackage[version=4]{mhchem}
  % If I'm loading the package to just define a few new commands I'll indent
  % two spaces right after loading the package and define the few new
  % commands here.  If I'm defining more than a few commands I usually do it
  % after loading all the packages.
  % Define "\nitrate" to be the chemical symbol for nitrate.
  \newcommand{\nitrate}{\ce{NO3{-}}}
  % Define "\pnitrate" (short for "parenthesized nitrate") to be the chemical
  % symbol for nitrate surrounded by parentheses.
  \newcommand{\pnitrate}{(\nitrate)}
  % "Define \vpnitrate" (short for "verbose parenthesized nitrate") to be
  % the word "nitrate" followed by a space followed by the chemical symbol
  % for nitrate with parentheses around it.
  \newcommand{\vpnitrate}{nitrate (\nitrate)}

% For
%     \cancel
%     \highlight
% See
%     http://ftp.math.purdue.edu/mirrors/ctan.org/macros/latex/contrib/siunitx/siunitx.pdf
% pages 11--12.
\usepackage{cancel}


% Redefine description, enumerate, and itemize lists.
% See
%     https://mirrors.concertpass.com/tex-archive/macros/latex/contrib/enumitem/enumitem.pdf
% \usepackage{enumitem}
% \setlist[itemize]{leftmargin=7pt,rightmargin=24pt}



% This gets rid of
%     [5] (./thesis.toc
%     ! Undefined control sequence.
%     \vbox_set:Nn ...box:D {\color_group_begin: #2\par
%                                                       \color_group_end: }
%     l.32 ...}Basic Circuit Components}{31}{section.67}
%                                                       %
%     ?
% and
%     [6]
%     ! Undefined control sequence.
%     \vbox_set:Nn ...box:D {\color_group_begin: #2\par
%                                                       \color_group_end: }
%     l.61 ...rline {P.1}Frenchspacing}{67}{section.445}
%                                                       %
%     ?
% errors.
% See
%     https://github.com/latex3/latex2e/issues/73
\usepackage{etoc}

% Define \setmaxprintline{number_of_columns}.
% \usepackage{hardwrap}

% For indexing.  Making an index is optional.
% Make these commands available:
%     COMMAND           DESCRIPTION
%     \index{string}    put "string" in index information
%     \makeindex        save information to make the index
%     \printindex       print the index
% See
%     https://ctan.org/pkg/makeidx?lang=en
% for more information.
\usepackage{makeidx}
  % By default \index ignores its argument.
  % This activates indexing.
  \makeindex
  % The "chapter name" for the index.
  \renewcommand{\indexname}{INDEX}

% The mathtools package
% (see http://mirror.utexas.edu/ctan/macros/latex/required/amsmath/amsmath.pdf)
% loads the amsmath package which defines the
%     align
%     align*
%     alignat
%     alignat*
%     equation
%     equation*
%     flalign
%     flalign*
%     gather
%     gather*
%     multitaper
%     multitaper*
%     split
% environments and extends amsmath by defining many other commands.
% See
%     https://ctan.org/pkg/amsmath
% for information about amsmath and
%     http://ctan.math.washington.edu/tex-archive/macros/latex/contrib/mathtools/mathtools.pdf
% for information about mathtools.
\usepackage{mathtools}

% Define \includemedia.
\usepackage{media9}

% Define \begin{multicols}{number_of_columns} ... \end{multicolumns}.
% Used in ap-text.tex.
\usepackage{multicol}

% Define \ditto.
\usepackage{pa-ditto}

% Define \FigureDash.
% \FigureDash is a dash the width of a digit in the current font.
\usepackage{pa-figure-dash}

% For PurdueThesis, PuTh, TeX, LaTeX, METAFONT, METAPOST, etc. related logos.
\usepackage{pa-logos}

% (Or maybe use isomath instead?  -mark  2021-06-20)
% Follow ISO 80000-2:2019
%     o   put e, i, j, and pi in upright font automatically
%     o   use, for example, "\di x" to get "\,mathrm{d}\/x"
% This loads
%     o   amsmath.sty (which is already loaded above)
%     o   mathtools.sty
%     o   upgreek.sty
% Load the package.
\usepackage{pa-mismath}
  % Tell mismath to put e, i, j, and pi in upright font automatically.
  \enumber
  \inumber
  \jnumber
  \pinumber
  % To typeset math italic e, i, j, and pi use
  %     \mathit e
  %     \mathit i
  %     \mathit j
  %     \itpi

% Define \MyRepeat{what}{repeat}.
% Do "what" "repeat" number of times.
\usepackage{pa-repeat}

% Define \FloatBarrier.
% \FloatBarrier process all unproccesed floats (tables, figures, etc.).
\usepackage{placeins}

% Define \hl.
% Undefine \st so soul will load without an error.
% I hope \st wasn't used for something important!
\let\st\relax
\usepackage{soul}

% Define \textcent.
\usepackage{textcomp}

% !!! This doesn't work yet, figure it out later.
% For \textprimstress.
% \usepackage{tipa}

% Needed for chapter "Graphics", section "TikZ and PGF".
\usepackage{tikz}
  % Needed to customize arrows.
  \usetikzlibrary{arrows.meta}
  % For electrical diagrams.
  % Uses the TikZ package.
  % The circuitikz name is short for "circuit TikZ".
  \usepackage{circuitikz}
  %
  \usepackage{menukeys}
  %
  % Needed for 3D TikZ stuff.
  \usetikzlibrary{3d}
  %
  % Needed for pa-typographic-conventions package.
  \usetikzlibrary{calc,shadows,shapes.misc,shapes.symbols}
  %
  % Needed for putting text along a path.
  \usetikzlibrary{decorations.text}
  %
  % Draw TikZ decorations.
  % Needed for at least the Kalman filter system model graphic.
  \usetikzlibrary{decorations.pathmorphing} % noisy shapes
  %
  % Fit shapes to coordinates.
  % Needed for at least the Kalman filter system model graphic.
  \usetikzlibrary{fit}
  %
  % Draw the background after the foreground.
  \usetikzlibrary{backgrounds}	% drawing the background after the foreground

% Needed for the Feynman diagram in ap-physics.tex.
% Tikz-feynman requires LuaLaTeX instead of pdflatex be run.
% LuaLaTeX screws up spacing in the list of figures so this
% is not loaded and LuaLaTeX should not be used.
\usepackage[compat=1.1.0]{tikz-feynman}


% The vertical space between a table heading and the table contents
% in a tabular environment.
\newcommand{\tabularspace}{\noalign{\vspace*{2pt}}}

% For \sfrac, used to do slanted fractions, similar to, e.g., 1/2,
% but 1 is small and high and 2 is small and low.
\usepackage{xfrac}

\usepackage{amsmath}
\usepackage{amssymb}
\usepackage{bbm}

\usepackage{lineno}
%\linenumbers


\newcommand{\ttbar}{\ensuremath{t\bar{t}}\xspace}
\renewcommand{\pp}{\ensuremath{pp}\xspace}
\newcommand{\beamenergy}{\ensuremath{\sqrt{s}=\SI{13}{\TeV}}\xspace}
\newcommand{\invfb}{\ensuremath{\si{\femto \b}^{-1}}\xspace}

\renewcommand{\ee}{\ensuremath{e^{+}e^{-}}\xspace}
\newcommand{\emu}{\ensuremath{e^{\pm}\mu^{\mp}}\xspace}
\newcommand{\mumu}{\ensuremath{\mu^{+}\mu^{-}}\xspace}

\newcommand{\metxy}{\ensuremath{E\!\!\!\!/_\text{x,y}}\xspace}
\newcommand{\pTmiss}{\ensuremath{\pT^\text{miss}}\xspace}
\newcommand{\ETmiss}{\ensuremath{E_{\mathrm{T}}^{\text{miss}}}\xspace}
\newcommand{\MET}{\ETmiss}
\newcommand{\pT}{\ensuremath{p_{\mathrm{T}}}\xspace}
\newcommand{\HT}{\ensuremath{H_{\mathrm{T}}}\xspace}

\newcommand{\mll}{\ensuremath{m_{\ell \bar{\ell}}}\xspace}


\newcommand{\Powheg}{{\textsc{Powhegv2}}}
\newcommand{\Pythia} {{\textsc{Pythia8}}}
\newcommand{\Geant} {{\textsc{Geant4}}}
\newcommand{\MadSpin} {{\textsc{MadSpin}}} 
\newcommand{\Herwig} {{\textsc{Herwig++}}} 
\newcommand{\MGaMCatNLO} {\textsc{MadGraph5\_aMc@NLO(FxFx)}}
\newcommand{\MGaMCatNLOOnly} {\textsc{MadGraph5\_aMc@NLO}}
\newcommand{\MGMLM} {\textsc{MadGraph5\_aMc@NLO(MLM)}}
\newcommand{\MG} {\textsc{MadGraph5(MLM)}}


\newcommand{\lumivalueSixPreVFP}{19.50\xspace \invfb} 
\newcommand{\lumierrSixPreVFP}{1.2\%}
\newcommand{\lumivalueSixPostVFP}{16.81\xspace \invfb} 
\newcommand{\lumierrSixPostVFP}{1.2\%}
\newcommand{\lumivalueSeven}{41.48\xspace \invfb} 
\newcommand{\lumierrSeven}{2.3\%}
\newcommand{\lumivalueEight}{59.83\xspace \invfb} 
\newcommand{\lumierrEight}{2.5\%}
\newcommand{\lumivalueRuniiUL}{137.7\xspace \invfb} 
\newcommand{\lumierrRuniiUL}{1.6\%}

\newcommand{\dy}{\ensuremath{Z/\gamma^*}}
\newcommand{\wjets}{\ensuremath{W+}jets} 
\newcommand{\zjets}{\ensuremath{Z+}jets} 
\newcommand{\tw}{\ensuremath{tW}} 

\newcommand{\xsecDYTenFifty}{22635.1}
\newcommand{\xsecDYFiftyInf}{6225.4}
\newcommand{\xsecTTBAR}{831.76}
\newcommand{\xsecTTBARdilept}{831.76\times0.10706}
\newcommand{\xsecTTBARljets}{831.76\times0.44113}
\newcommand{\xsecTTBARhadronic}{831.76\times0.45441}
\newcommand{\xsecSINGLETOPtw}{35.85\times0.54559}
\newcommand{\xsecWlnu}{61526.7}
\newcommand{\xsecWW}{118.7}
\newcommand{\xsecWZ}{47.13}
\newcommand{\xsecZZ}{16.523}
\newcommand{\xsecTTWJETSlnu}{0.2043}
\newcommand{\xsecTTWJETSqq}{0.4062}
\newcommand{\xsecTTZllnunu}{0.2529}
\newcommand{\xsecTTZqq}{0.5297}

\newcommand{\mtt} {\ensuremath{m_{\ttbar}}\xspace}
\newcommand{\ytt} {\ensuremath{y_{\ttbar}}\xspace}
\newcommand{\pttt} {\ensuremath{\pT^{\ttbar}}\xspace}
\newcommand{\yt} {\ensuremath{y_t}\xspace}
\newcommand{\ptt} {\ensuremath{\pT^t}\xspace}

\newcommand{\matr}[1]{\mathbf{#1}}

% Define \I.
% \I1 does \indent once, \I2 does \indent twice, etc.
\newcommand{\I}[1]{\MyRepeat{\indent}{#1}}

% Define \MyI.
% Typeset my input.
\long\def\MyI#1%
  {%
    {%
      \fontsize{8}{10}\tt
      \VerbatimInput
        [
          firstnumber = 1,
          numbers     = left,
          xleftmargin = 0.33in,
        ]%
        {#1}
    }%
  }

% Define \MyIO.
% Typeset my input and output.
% The input will all be on the same page.
% The output may be split over multiple pages.
\newcommand{\MyIO}
  {%
    \input{z.out}

    {%
      \fontsize{8}{10}\tt
      \VerbatimInput
        [
          firstnumber = 1,
          numbers     = left,
          xleftmargin = 0.33in,
        ]
        {z.out}
    }
    \FloatBarrier
  }

% Define \NL (newline) so LaTeX goes to the next output line.
% Just doing \\ complains
%     ! LaTeX Error: There's no line here to end.
% \mbox{} is an empty math box.
\newcommand{\NL}{\mbox{}\\}

% Print a list of files used and their version numbers in the log file.
\listfiles


% \def\bibindent{0em}
% Customize the bibliography.
% \DefineBibliographyStrings{english}{
%   urlfrom = {URLFROM},
%   urlseen = {URLSEEN}
% }

% For typographical conventions stuff including
%     \Emph{...}
%     \First{...}
%     \Keys{...}
%     \Literal{...}
%     \Menu{...}
%     \Place{...}
%     \Shell{...}
% This must be after
%     \usepackage{tikz}
\usepackage{pa-typographic-conventions}


% For the \begin{example} ... \end{example} environment
% used in ap-linguistics.tex.
\usepackage{covington}
\usepackage{slgloss}

% "CTAN---Comprehensive" did not get hyphenated and extended
% into the right margin when using BibLaTeX and the apa style.
% These did not change it:
%     \hyphenation{Com-pre-hen-sive}
%     \hyphenation{CTAN---Com-pre-hen-sive}
% I changed    publisher = {CTAN---Comprehensive TeX Archive Network},
% to           publisher = {CTAN---Com\-pre\-hen\-sive TeX Archive Network},
% in my all-biblatex.bib file and it worked as expeceted.
% If you need to change the hyphenation points of a word in the text
% you can do, for example,
%     \hyphenation{ve-ry-od-dly-hy-phen-at-ed}


\begin{document}

\setcounter{tocdepth}{3}

\maketitle

% Define front matter
%     dedication
%     acknowledgments
%     preface
%     table of contents
%     list of tables
%     list of figures
%     list of symbols
%     abbreviations
%     nomenclature
%     glossary
%     abstract
\ProvidesFile{ch-Front.tex}[2022-10-05 front matter chapter]
%
%  This is the ``front matter'' for the thesis.
%
%  REFERENCES
%
%    TCMOS17
%      The Chicago Manual of Style Online, 17th edition.
%      https://www.chicagomanualofstyle.org/home.html
%      retrieved on 2020-02-29
%
%    TEMPL
%      Thesis and Disertation Office Templates.
%      https://www.purdue.edu/gradschool/research/thesis/templates.html
%      retrieved on 2020-02-29
%
%    WNNCD
%    Webster's Ninth New Collegiate Dictionary.
%

%
%   Only Purdue University uses this page
%
%   Comment out \begin{statement} through \end{statement}
%   if you are not at Purdue University.
%
% Statement of Thesis/Dissertation Approval Page
% This page is REQUIRED.  The page should be numbered "2"
% and should NOT be listed in your TABLE OF CONTENTS.
\begin{statement}
  % Delete or add \entry commands as needed for all committe members.
  \entry{Dr.~Andreas Jung, Chair}{Department of Physics and Astronomy}
  \entry{Dr.~John Finley}{Department of Physics and Astronomy}
  \entry{Dr.~Matthew Jones}{Department of Physics and Astronomy}
  \entry{Dr.~Martin Kruczenski}{Department of Physics and Astronomy}

  % There should be one \approvedby command containing the
  % "FORM 9 THESIS FORM HEAD NAME HERE" (from TEMPL, retrieved on 2020-03-01).
  \approvedby{Dr.~Gabor Csathy}{\centering Head of the Department of Physics and Astronomy\par}
\end{statement}

% Dedication page is optional.
% A name and often a message in tribute to a person or cause.
% References: WEB9 332.
%\begin{dedication}
%  To graduate students
%\end{dedication}

% Acknowledgements page is optional but most theses include
% a brief statement of appreciation or recognition of special
% assistance.
\begin{acknowledgments}
Acknowledgements Here
\end{acknowledgments}

% The preface is optional.
% References: TCMOS17 1.49, WEB9 927.
%\begin{preface}
%
%\end{preface}

% The Table of Contents is required.
% The Table of Contents will be automatically created for you
% using information you supply in
%     \chapter
%     \section
%     \subsection
%     \subsubsection
%     commands.
\pdfbookmark{TABLE OF CONTENTS}{Contents}
\tableofcontents

% If your thesis has tables, a list of tables is required.
% The List of Tables will be automatically created for you using
% information you supply in
%     \begin{table} ... \end{table}
% environments.
\listoftables

% If your thesis has figures, a list of figures is required.
% The List of Figures will be automatically created for you using
% information you supply in
%     \begin{figure} ... \end{figure}
% environments.
\listoffigures

% If your thesis has listings, a list of listings is required.
% The List of Listings will be automatically created for you using
% information you supply in
%     \begin{ZZlisting} ... \end{ZZlisting}
% environments.
%\ZZlistoflistings

% If your thesis has protocols, you may want to do a list of protocols.
% The List of Protocols will be automatically created for you using
% information you supply in
%     \begin{protocol} ... \end{protocol}
% environments.
%\listofprotocols

% If your thesis has schemes, you may want to do a list of schemes.
% The List of Schemes will be automatically created for you using
% information you supply in
%     \begin{scheme} ... \end{scheme}
% environments.
%\listofschemes

% List of Symbols is optional.
%\begin{symbols}
%  $+$& Positive Electric Charge\cr
%  $-$& Negative Electric Charge\cr  
%  $p$& Proton\cr
%  $n$& Neutron\cr
%  $\ell$& Lepton\cr
%  $e$& Electron\cr
%  $\mu$& Muon\cr
%  $\tau$ & Tau\cr
%  $\nu$& Neutrino\cr
%  $\gamma$& Photon\cr
%  $Z$& Z Boson\cr
%  $W^\pm$& W Boson\cr
%  $q$& Quark\cr
%  $u$& Up Quark\cr
%  $d$& Down Quark\cr
%  $c$& Charm Quark\cr
%  $s$& Strange Quark\cr
%  $t$& Top Quark\cr
%  $b$& Bottom Quark\cr
%  $g$& Gluon\cr
%  $R$& Red Color Charge\cr
%  $G$& Green Color Charge\cr
%  $B$& Blue Color Charge\cr
%  $H$& Higgs Boson\cr
%  $\bar{x}$& Anti-matter counterpart for arbitrary particle $x$\cr
%  $E$& Energy\cr
%  $m$& Mass\cr
%  $\pT$& Transverse Momentum\cr
%  $\eta$& Pseudorapidity\cr
%  $y$& Rapidity\cr
%  $\phi$& Azimuthal Angle\cr
%  $\MET$& Missing Transverse Energy\cr
%  $\Delta R$& Angular Separation in $\eta$-$\phi$ Space\cr
%  $\sqrt{s}$& Center-of-Mass Energy\cr
%  $\mathcal{L}$& Luminosity\cr
%  $\sigma$& Cross-section\cr
%  $N$& Number of Events\cr
%  $w$& Weight\cr
%  $SF$& Scale Factor\cr
%  & \cr
%  & \cr
%  & \cr
%  & \cr
%\end{symbols}

% List of Abbreviations is optional.
%\begin{abbreviations}
%  3DIC& 3-Dimensional Integrated Circuit\cr
%  BEH& Brout-Englert-Higgs\cr
%  BSM& Beyond the Standard Model\cr
%  CERN& European Center for Nuclear Research\cr
%  CHS& Charged-Hadron Subtraction\cr
%  CKM& Cabibbo-Kobayashi-Maskawa\cr
%  CMS& Compact Muon Solenoid\cr
%  CR& Color Reconnection\cr
%  DBI& Direct Bonding Interconnect\cr
%  DY& Drell-Yan\cr
%  EB& ECAL Barrel\cr
%  ECAL& Electromagnetic Calorimeter\cr
%  EDM& Event Data Model\cr
%  EE& ECAL End-cap\cr
%  EWSB& Electroweak Symmetry Breaking\cr
%  FEA& Finite Element Analysis\cr
%  FSR& Final State Radiation\cr
%  FNAL& Fermi National Accelerator Laborator\cr
%  GR& General Relativity\cr
%  GSF& Gaussian Sum Filter\cr
%  GUT& Grand Unification Theory\cr
%  GWS& Glashow-Weinberg-Salam\cr
%  HB& HCAL Barrel\cr
%  HCAL& Hadronic Calorimeter\cr
%  HE& HCAL End-cap\cr
%  HIPM& Heavily Ionizing Particle Mitigation\cr
%  IP& Interaction Point\cr
%  ISR& Initial State Radiation\cr
%  HL-LHC& High Luminosity Large Hadron Collider\cr
%  JEC& Jet Energy Correction\cr
%  JER& Jet Energy Resolution\cr
%  JES& Jet Energy Scale\cr
%  KF& Kalman Filter\cr
%  LGAD& Low Gain Avalanche Diode\cr
%  LHC& Large Hadron Collider\cr
%  LO& Leading-Order\cr
%  MC& Monte Carlo\cr
%  ME& Matrix Element\cr
%  MET& Missing Transverse Energy\cr
%  MIP& Minimum Ionizing Particle\cr
%  MPI& Multiple Parton Interactions\cr
%  MTD& MIP Timing Detector\cr
%  NLO& Next-to-Leading-Order\cr
%  NNLL& Next-to-Next-to-Leading-Logarithmic\cr
%  NNLO& Next-to-Next-to-Leading-Order\cr
%  NP& New Physics\cr
%  PDF& Parton Distribution Function\cr
%  PDG& Particle Data Group\cr  
%  PF& Particle-Flow\cr
%  PU& Pileup\cr
%  QED& Quantum Electrodynamics\cr  
%  QFT& Quantum Field Theory\cr
%  SF& Scale Factor\cr
%  SiPM& Silicon Photomultiplier\cr
%  SM& Standard Model of Particle Physics\cr
%  SMEFT& Standard Model Effective Field Theory\cr
%  SPICE& Simulation Program with Integrated Circuit Emphasis\cr
%  TCAD& Technology Computer-Aided Design\cr
%  TSV& Through-Silicon Via\cr
%  UE& Underlying Event\cr
%  WIMP& Weakly Interacting Massive Particle\cr
%  ZMF& Zero Momentum Frame\cr 
%  & \cr
%  & \cr
%  & \cr
%  & \cr
%  & \cr
%\end{abbreviations}


% Abstract is required.
% Note that the information for the first paragraph of the output
% doesn't need to be input here...it is put in automatically from
% information you supplied earlier using \title, \author, \degree,
% and \majorprof.
% Reference: PU 17.
\begin{abstract}%

Abstract Here

\end{abstract}



%
% Put chapter \include commands here.
%
\ProvidesFile{chapters/ch-Summary.tex}

\chapter{SUMMARY AND OUTLOOK}
\label{Conclusion}

Citation so bibliography works \cite{Chatrchyan:1129810}

% Summary and/or conclusions are optional but often used.
% The summary and/or conclusions often are the last
% the last major division(s) of the text.
% Reference: TM2017 page 32.




% Appendices are optional.  Not all theses contain appendices.
% An appendix is used for supplementary illustrative material,
% original data, computer programs, and other material that is not
% necessarily appropriate for inclusion within the text of your
% thesis.
% Reference: TM2017 page 33.
%
% Use ``\appendix'' for one appendix or ``\appendices'' for more than
% one appendix.
\appendices



% \immediate\setlength{\bibhang}{-3in}
% \immediate\setlength{\itemindent}{3in}
% \immediate\setlength{\rightmargin}{3in}

%
% This is only done if you are using BibLaTeX.
%
\makeatletter  % commented out on 2022-01-26
  \defbibenvironment{bibliography}
    {%
      \list
        {%
          \printtext[labelnumberwidth]%
          {%
            \printfield{prefixnumber}%
            \printfield{labelnumber}%
          }%
        }%
        {%
          \setlength{\bibhang}{1in} %%%%% was 0pt
          \setlength{\itemindent}{1in}%  -\leftmargin} %%%%% was 0pt
          \setlength{\itemsep}{\bibitemsep}%
          \setlength{\leftmargin}{0pt}%  .22in} % 0.42in}
          \setlength{\parsep}{\bibparsep}%
           \setlength{\rightmargin}{0.33in}%
        }%
    }
    {\endlist}
    {\item}
\makeatother  % commented out on 2022-01-26

% \immediate\setlength{\labelnumberwidth}{1.5in} %%%%% was commented out
\setlength{\labelwidth}{1.5in}
\def\sllnsez{[999] }

{%
  % Make _ in URLs visible.
  % \def\t{\char'137}%
  \catcode`*=\active
  \def*{\char'137}%  \char'137 is _
  \PrintBibliography
}


% My filename conventions:
%     FILE THAT START WITH    ARE
%     ap-                     appendices
%     ch-                     chapters
%     gr-                     graphics
%     pa-                     packages
%     z                       temporary files

% LaTeX won't read after the \end{document} command.
% You can put notes to yourself or LaTeX input not
% ready for use after "\end{document}" if you'd like.
\end{document}
